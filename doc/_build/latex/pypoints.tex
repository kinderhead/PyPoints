%% Generated by Sphinx.
\def\sphinxdocclass{report}
\documentclass[letterpaper,10pt,english]{sphinxmanual}
\ifdefined\pdfpxdimen
   \let\sphinxpxdimen\pdfpxdimen\else\newdimen\sphinxpxdimen
\fi \sphinxpxdimen=.75bp\relax

\PassOptionsToPackage{warn}{textcomp}
\usepackage[utf8]{inputenc}
\ifdefined\DeclareUnicodeCharacter
% support both utf8 and utf8x syntaxes
  \ifdefined\DeclareUnicodeCharacterAsOptional
    \def\sphinxDUC#1{\DeclareUnicodeCharacter{"#1}}
  \else
    \let\sphinxDUC\DeclareUnicodeCharacter
  \fi
  \sphinxDUC{00A0}{\nobreakspace}
  \sphinxDUC{2500}{\sphinxunichar{2500}}
  \sphinxDUC{2502}{\sphinxunichar{2502}}
  \sphinxDUC{2514}{\sphinxunichar{2514}}
  \sphinxDUC{251C}{\sphinxunichar{251C}}
  \sphinxDUC{2572}{\textbackslash}
\fi
\usepackage{cmap}
\usepackage[T1]{fontenc}
\usepackage{amsmath,amssymb,amstext}
\usepackage{babel}



\usepackage{times}
\expandafter\ifx\csname T@LGR\endcsname\relax
\else
% LGR was declared as font encoding
  \substitutefont{LGR}{\rmdefault}{cmr}
  \substitutefont{LGR}{\sfdefault}{cmss}
  \substitutefont{LGR}{\ttdefault}{cmtt}
\fi
\expandafter\ifx\csname T@X2\endcsname\relax
  \expandafter\ifx\csname T@T2A\endcsname\relax
  \else
  % T2A was declared as font encoding
    \substitutefont{T2A}{\rmdefault}{cmr}
    \substitutefont{T2A}{\sfdefault}{cmss}
    \substitutefont{T2A}{\ttdefault}{cmtt}
  \fi
\else
% X2 was declared as font encoding
  \substitutefont{X2}{\rmdefault}{cmr}
  \substitutefont{X2}{\sfdefault}{cmss}
  \substitutefont{X2}{\ttdefault}{cmtt}
\fi


\usepackage[Bjarne]{fncychap}
\usepackage{sphinx}

\fvset{fontsize=\small}
\usepackage{geometry}

% Include hyperref last.
\usepackage{hyperref}
% Fix anchor placement for figures with captions.
\usepackage{hypcap}% it must be loaded after hyperref.
% Set up styles of URL: it should be placed after hyperref.
\urlstyle{same}
\addto\captionsenglish{\renewcommand{\contentsname}{Contents:}}

\usepackage{sphinxmessages}
\setcounter{tocdepth}{1}



\title{PyPoints}
\date{Apr 17, 2020}
\release{0.1}
\author{Daniel}
\newcommand{\sphinxlogo}{\vbox{}}
\renewcommand{\releasename}{Release}
\makeindex
\begin{document}

\pagestyle{empty}
\sphinxmaketitle
\pagestyle{plain}
\sphinxtableofcontents
\pagestyle{normal}
\phantomsection\label{\detokenize{index::doc}}



\chapter{PyPoints}
\label{\detokenize{pypoints:module-pypoints}}\label{\detokenize{pypoints:pypoints}}\label{\detokenize{pypoints::doc}}\index{pypoints (module)@\spxentry{pypoints}\spxextra{module}}
A curses-based python module. Each character on the terminal is a point.
This module give you tools to use to build these points
\index{Blueprint (class in pypoints)@\spxentry{Blueprint}\spxextra{class in pypoints}}

\begin{fulllineitems}
\phantomsection\label{\detokenize{pypoints:pypoints.Blueprint}}\pysiglinewithargsret{\sphinxbfcode{\sphinxupquote{class }}\sphinxcode{\sphinxupquote{pypoints.}}\sphinxbfcode{\sphinxupquote{Blueprint}}}{\emph{file}, \emph{nofile=False}}{}
Bases: \sphinxcode{\sphinxupquote{object}}

Allows you to draw custom objects and fully customize objects like {\hyperref[\detokenize{pypoints:pypoints.MenuBox}]{\sphinxcrossref{\sphinxcode{\sphinxupquote{MenuBox}}}}} and {\hyperref[\detokenize{pypoints:pypoints.Text}]{\sphinxcrossref{\sphinxcode{\sphinxupquote{Text}}}}}. It uses a json file or string to load the Blueprint. Can drawn with {\hyperref[\detokenize{pypoints:pypoints.Shape}]{\sphinxcrossref{\sphinxcode{\sphinxupquote{Shape}}}}}
\begin{quote}\begin{description}
\item[{Parameters}] \leavevmode\begin{itemize}
\item {} 
\sphinxstyleliteralstrong{\sphinxupquote{file}} (\sphinxstyleliteralemphasis{\sphinxupquote{string}}) \textendash{} The file name of the file. If \sphinxcode{\sphinxupquote{nofile}} is \sphinxcode{\sphinxupquote{True}}, then it is the json string

\item {} 
\sphinxstyleliteralstrong{\sphinxupquote{nofile}} (\sphinxstyleliteralemphasis{\sphinxupquote{bool}}) \textendash{} (default \sphinxcode{\sphinxupquote{False}}) If \sphinxcode{\sphinxupquote{True}}, \sphinxcode{\sphinxupquote{file}} is interpreted as a string

\end{itemize}

\end{description}\end{quote}

\end{fulllineitems}

\index{Color (class in pypoints)@\spxentry{Color}\spxextra{class in pypoints}}

\begin{fulllineitems}
\phantomsection\label{\detokenize{pypoints:pypoints.Color}}\pysiglinewithargsret{\sphinxbfcode{\sphinxupquote{class }}\sphinxcode{\sphinxupquote{pypoints.}}\sphinxbfcode{\sphinxupquote{Color}}}{\emph{fg}, \emph{bg}}{}
Bases: \sphinxcode{\sphinxupquote{object}}

A color definition used for {\hyperref[\detokenize{pypoints:pypoints.Font}]{\sphinxcrossref{\sphinxcode{\sphinxupquote{Font}}}}}. PyPoints has variables that you can use for this. The colors are black, blue, cyan, green, magenta, red, white, and yellow
\begin{quote}\begin{description}
\item[{Parameters}] \leavevmode\begin{itemize}
\item {} 
\sphinxstyleliteralstrong{\sphinxupquote{fg}} (\sphinxstyleliteralemphasis{\sphinxupquote{byte}}) \textendash{} The foreground color

\item {} 
\sphinxstyleliteralstrong{\sphinxupquote{bg}} (\sphinxstyleliteralemphasis{\sphinxupquote{byte}}) \textendash{} The background color

\end{itemize}

\item[{Example}] \leavevmode
\end{description}\end{quote}

\begin{sphinxVerbatim}[commandchars=\\\{\}]
\PYG{n}{color} \PYG{o}{=} \PYG{n}{Color}\PYG{p}{(}\PYG{n}{black}\PYG{p}{,} \PYG{n}{yellow}\PYG{p}{)}
\end{sphinxVerbatim}


\sphinxstrong{See also:}


{\hyperref[\detokenize{pypoints:pypoints.Font}]{\sphinxcrossref{\sphinxcode{\sphinxupquote{Font}}}}}



\end{fulllineitems}

\index{ColorGet (class in pypoints)@\spxentry{ColorGet}\spxextra{class in pypoints}}

\begin{fulllineitems}
\phantomsection\label{\detokenize{pypoints:pypoints.ColorGet}}\pysigline{\sphinxbfcode{\sphinxupquote{class }}\sphinxcode{\sphinxupquote{pypoints.}}\sphinxbfcode{\sphinxupquote{ColorGet}}}
Bases: \sphinxcode{\sphinxupquote{object}}

A PyPoints registry class that is used to register curses color numbers.

\begin{sphinxadmonition}{warning}{Warning:}
Do not use this class. It is an internal usage class only
\end{sphinxadmonition}
\index{get() (pypoints.ColorGet method)@\spxentry{get()}\spxextra{pypoints.ColorGet method}}

\begin{fulllineitems}
\phantomsection\label{\detokenize{pypoints:pypoints.ColorGet.get}}\pysiglinewithargsret{\sphinxbfcode{\sphinxupquote{get}}}{}{}
\end{fulllineitems}


\end{fulllineitems}

\index{Font (class in pypoints)@\spxentry{Font}\spxextra{class in pypoints}}

\begin{fulllineitems}
\phantomsection\label{\detokenize{pypoints:pypoints.Font}}\pysiglinewithargsret{\sphinxbfcode{\sphinxupquote{class }}\sphinxcode{\sphinxupquote{pypoints.}}\sphinxbfcode{\sphinxupquote{Font}}}{\emph{color}, \emph{extra=None}}{}
Bases: \sphinxcode{\sphinxupquote{object}}

Used to make a {\hyperref[\detokenize{pypoints:pypoints.Point}]{\sphinxcrossref{\sphinxcode{\sphinxupquote{Point}}}}} with custom colors and display
\begin{quote}\begin{description}
\item[{Parameters}] \leavevmode\begin{itemize}
\item {} 
\sphinxstyleliteralstrong{\sphinxupquote{color}} ({\hyperref[\detokenize{pypoints:pypoints.Color}]{\sphinxcrossref{\sphinxcode{\sphinxupquote{Color}}}}}) \textendash{} The color of the text

\item {} 
\sphinxstyleliteralstrong{\sphinxupquote{extra}} (\sphinxstyleliteralemphasis{\sphinxupquote{byte}}) \textendash{} (optional) Extra text display types. PyPoints supports bold, dim, reverse, standout, and underline. These types are available as variables in the module

\end{itemize}

\item[{Example}] \leavevmode
\end{description}\end{quote}

\begin{sphinxVerbatim}[commandchars=\\\{\}]
\PYG{n}{color} \PYG{o}{=} \PYG{n}{Color}\PYG{p}{(}\PYG{n}{green}\PYG{p}{,} \PYG{n}{blue}\PYG{p}{)}
\PYG{n}{font} \PYG{o}{=} \PYG{n}{Font}\PYG{p}{(}\PYG{n}{color}\PYG{p}{,} \PYG{n}{bold}\PYG{p}{)}
\PYG{n}{point} \PYG{o}{=} \PYG{n}{Point}\PYG{p}{(}\PYG{l+s+s2}{\PYGZdq{}}\PYG{l+s+s2}{a}\PYG{l+s+s2}{\PYGZdq{}}\PYG{p}{,} \PYG{l+m+mi}{0}\PYG{p}{,} \PYG{l+m+mi}{0}\PYG{p}{,} \PYG{l+m+mi}{0}\PYG{p}{,} \PYG{n}{font}\PYG{p}{)}
\end{sphinxVerbatim}

\begin{sphinxadmonition}{warning}{Warning:}
Do not change \sphinxcode{\sphinxupquote{font.regid}}. Blueprints using that font will fail. Keep in mind that the order of the font definition determines its regid. If using blueprints, don’t change the order of your fonts
\end{sphinxadmonition}


\sphinxstrong{See also:}


{\hyperref[\detokenize{pypoints:pypoints.Color}]{\sphinxcrossref{\sphinxcode{\sphinxupquote{Color}}}}}, {\hyperref[\detokenize{pypoints:pypoints.Point}]{\sphinxcrossref{\sphinxcode{\sphinxupquote{Point}}}}}, {\hyperref[\detokenize{pypoints:pypoints.Blueprint}]{\sphinxcrossref{\sphinxcode{\sphinxupquote{Blueprint}}}}}


\index{export() (pypoints.Font method)@\spxentry{export()}\spxextra{pypoints.Font method}}

\begin{fulllineitems}
\phantomsection\label{\detokenize{pypoints:pypoints.Font.export}}\pysiglinewithargsret{\sphinxbfcode{\sphinxupquote{export}}}{}{}
Exports font to file “font\textless{}regid\textgreater{}.pk” as a pickle file. It can be unpickled with the pickle module.
\begin{quote}\begin{description}
\item[{Example}] \leavevmode
\end{description}\end{quote}

\begin{sphinxVerbatim}[commandchars=\\\{\}]
\PYG{n}{font}\PYG{o}{.}\PYG{n}{export}\PYG{p}{(}\PYG{p}{)}
\end{sphinxVerbatim}

\begin{sphinxadmonition}{note}{Note:}
To get regid, use \sphinxcode{\sphinxupquote{font.regid}}, or look at the order of your fonts. First one gets 0, and so on
\end{sphinxadmonition}


\sphinxstrong{See also:}


{\hyperref[\detokenize{pypoints:pypoints.FontRegistry}]{\sphinxcrossref{\sphinxcode{\sphinxupquote{FontRegistry}}}}}



\end{fulllineitems}


\end{fulllineitems}

\index{FontRegistry (class in pypoints)@\spxentry{FontRegistry}\spxextra{class in pypoints}}

\begin{fulllineitems}
\phantomsection\label{\detokenize{pypoints:pypoints.FontRegistry}}\pysigline{\sphinxbfcode{\sphinxupquote{class }}\sphinxcode{\sphinxupquote{pypoints.}}\sphinxbfcode{\sphinxupquote{FontRegistry}}}
Bases: \sphinxcode{\sphinxupquote{object}}

A PyPoints registry class that is used to register {\hyperref[\detokenize{pypoints:pypoints.Font}]{\sphinxcrossref{\sphinxcode{\sphinxupquote{Font}}}}}

\begin{sphinxadmonition}{warning}{Warning:}
Do not use this class. It is an internal usage class only
\end{sphinxadmonition}
\index{get() (pypoints.FontRegistry method)@\spxentry{get()}\spxextra{pypoints.FontRegistry method}}

\begin{fulllineitems}
\phantomsection\label{\detokenize{pypoints:pypoints.FontRegistry.get}}\pysiglinewithargsret{\sphinxbfcode{\sphinxupquote{get}}}{\emph{font}}{}
\end{fulllineitems}

\index{register() (pypoints.FontRegistry method)@\spxentry{register()}\spxextra{pypoints.FontRegistry method}}

\begin{fulllineitems}
\phantomsection\label{\detokenize{pypoints:pypoints.FontRegistry.register}}\pysiglinewithargsret{\sphinxbfcode{\sphinxupquote{register}}}{\emph{font}}{}
\end{fulllineitems}


\end{fulllineitems}

\index{HLine (class in pypoints)@\spxentry{HLine}\spxextra{class in pypoints}}

\begin{fulllineitems}
\phantomsection\label{\detokenize{pypoints:pypoints.HLine}}\pysiglinewithargsret{\sphinxbfcode{\sphinxupquote{class }}\sphinxcode{\sphinxupquote{pypoints.}}\sphinxbfcode{\sphinxupquote{HLine}}}{\emph{sx}, \emph{ex}, \emph{y}, \emph{char}, \emph{cfield}, \emph{font=None}}{}
Bases: \sphinxcode{\sphinxupquote{object}}

Makes a horizontal line
\begin{quote}\begin{description}
\item[{Parameters}] \leavevmode\begin{itemize}
\item {} 
\sphinxstyleliteralstrong{\sphinxupquote{sx}} (\sphinxstyleliteralemphasis{\sphinxupquote{int}}) \textendash{} Start \sphinxcode{\sphinxupquote{x}}

\item {} 
\sphinxstyleliteralstrong{\sphinxupquote{ex}} (\sphinxstyleliteralemphasis{\sphinxupquote{int}}) \textendash{} End \sphinxcode{\sphinxupquote{x}}

\item {} 
\sphinxstyleliteralstrong{\sphinxupquote{y}} (\sphinxstyleliteralemphasis{\sphinxupquote{int}}) \textendash{} \sphinxcode{\sphinxupquote{y}} position

\item {} 
\sphinxstyleliteralstrong{\sphinxupquote{char}} (\sphinxstyleliteralemphasis{\sphinxupquote{string}}) \textendash{} The character that the line consists of

\item {} 
\sphinxstyleliteralstrong{\sphinxupquote{cfield}} (\sphinxstyleliteralemphasis{\sphinxupquote{int}}) \textendash{} The field the line appears

\item {} 
\sphinxstyleliteralstrong{\sphinxupquote{font}} ({\hyperref[\detokenize{pypoints:pypoints.Font}]{\sphinxcrossref{\sphinxcode{\sphinxupquote{Font}}}}}) \textendash{} (optional) The font of the points in the line

\end{itemize}

\item[{Example}] \leavevmode
\end{description}\end{quote}

\begin{sphinxVerbatim}[commandchars=\\\{\}]
\PYG{n}{HLine}\PYG{p}{(}\PYG{l+m+mi}{1}\PYG{p}{,} \PYG{l+m+mi}{7}\PYG{p}{,} \PYG{l+m+mi}{1}\PYG{p}{,} \PYG{l+s+s2}{\PYGZdq{}}\PYG{l+s+s2}{\PYGZhy{}}\PYG{l+s+s2}{\PYGZdq{}}\PYG{p}{,} \PYG{l+m+mi}{0}\PYG{p}{)}
\end{sphinxVerbatim}


\sphinxstrong{See also:}


{\hyperref[\detokenize{pypoints:pypoints.VLine}]{\sphinxcrossref{\sphinxcode{\sphinxupquote{VLine}}}}}


\index{build() (pypoints.HLine method)@\spxentry{build()}\spxextra{pypoints.HLine method}}

\begin{fulllineitems}
\phantomsection\label{\detokenize{pypoints:pypoints.HLine.build}}\pysiglinewithargsret{\sphinxbfcode{\sphinxupquote{build}}}{}{}
There is no need to use this method. It is run automatically

\end{fulllineitems}

\index{remove() (pypoints.HLine method)@\spxentry{remove()}\spxextra{pypoints.HLine method}}

\begin{fulllineitems}
\phantomsection\label{\detokenize{pypoints:pypoints.HLine.remove}}\pysiglinewithargsret{\sphinxbfcode{\sphinxupquote{remove}}}{\emph{kill=False}}{}
Removes all points in the line
\begin{quote}\begin{description}
\item[{Parameters}] \leavevmode
\sphinxstyleliteralstrong{\sphinxupquote{kill}} (\sphinxstyleliteralemphasis{\sphinxupquote{bool}}) \textendash{} (optional) (default \sphinxcode{\sphinxupquote{False}}) Deletes the line. Any subsequent calls or references to the line will fail because the line no longer exists.

\end{description}\end{quote}

\begin{sphinxadmonition}{warning}{Warning:}
This action is irreversable
\end{sphinxadmonition}


\sphinxstrong{See also:}


\sphinxcode{\sphinxupquote{Point.}}{\hyperref[\detokenize{pypoints:pypoints.Point.remove}]{\sphinxcrossref{\sphinxcode{\sphinxupquote{remove()}}}}}



\end{fulllineitems}


\end{fulllineitems}

\index{IncompatibleBlueprintType() (in module pypoints)@\spxentry{IncompatibleBlueprintType()}\spxextra{in module pypoints}}

\begin{fulllineitems}
\phantomsection\label{\detokenize{pypoints:pypoints.IncompatibleBlueprintType}}\pysiglinewithargsret{\sphinxcode{\sphinxupquote{pypoints.}}\sphinxbfcode{\sphinxupquote{IncompatibleBlueprintType}}}{\emph{Exception}}{}
An error that is called when the wrong blueprint is given to a
blueprint using function


\sphinxstrong{See also:}


{\hyperref[\detokenize{pypoints:pypoints.Blueprint}]{\sphinxcrossref{\sphinxcode{\sphinxupquote{Blueprint}}}}}, {\hyperref[\detokenize{pypoints:pypoints.Shape}]{\sphinxcrossref{\sphinxcode{\sphinxupquote{Shape}}}}}



\end{fulllineitems}

\index{MenuBox (class in pypoints)@\spxentry{MenuBox}\spxextra{class in pypoints}}

\begin{fulllineitems}
\phantomsection\label{\detokenize{pypoints:pypoints.MenuBox}}\pysiglinewithargsret{\sphinxbfcode{\sphinxupquote{class }}\sphinxcode{\sphinxupquote{pypoints.}}\sphinxbfcode{\sphinxupquote{MenuBox}}}{\emph{x}, \emph{y}, \emph{opts}, \emph{cfield}, \emph{font=None}, \emph{blueprint=None}}{}
Bases: \sphinxcode{\sphinxupquote{object}}
\index{draw() (pypoints.MenuBox method)@\spxentry{draw()}\spxextra{pypoints.MenuBox method}}

\begin{fulllineitems}
\phantomsection\label{\detokenize{pypoints:pypoints.MenuBox.draw}}\pysiglinewithargsret{\sphinxbfcode{\sphinxupquote{draw}}}{}{}
\end{fulllineitems}


\end{fulllineitems}

\index{Point (class in pypoints)@\spxentry{Point}\spxextra{class in pypoints}}

\begin{fulllineitems}
\phantomsection\label{\detokenize{pypoints:pypoints.Point}}\pysiglinewithargsret{\sphinxbfcode{\sphinxupquote{class }}\sphinxcode{\sphinxupquote{pypoints.}}\sphinxbfcode{\sphinxupquote{Point}}}{\emph{char}, \emph{x}, \emph{y}, \emph{cfield}, \emph{font=None}, \emph{active=True}}{}
Bases: \sphinxcode{\sphinxupquote{object}}

The core foundation of the PyPoints module. It is a single character on the terminal window.
\begin{quote}\begin{description}
\item[{Parameters}] \leavevmode\begin{itemize}
\item {} 
\sphinxstyleliteralstrong{\sphinxupquote{char}} (\sphinxstyleliteralemphasis{\sphinxupquote{string}}) \textendash{} The character of the point.

\item {} 
\sphinxstyleliteralstrong{\sphinxupquote{x}} (\sphinxstyleliteralemphasis{\sphinxupquote{int}}) \textendash{} The \sphinxcode{\sphinxupquote{x}} position on the window

\item {} 
\sphinxstyleliteralstrong{\sphinxupquote{y}} (\sphinxstyleliteralemphasis{\sphinxupquote{int}}) \textendash{} The \sphinxcode{\sphinxupquote{y}} position on the window

\item {} 
\sphinxstyleliteralstrong{\sphinxupquote{cfield}} (\sphinxstyleliteralemphasis{\sphinxupquote{int}}) \textendash{} The field that the point is displayed on

\item {} 
\sphinxstyleliteralstrong{\sphinxupquote{font}} ({\hyperref[\detokenize{pypoints:pypoints.Font}]{\sphinxcrossref{\sphinxcode{\sphinxupquote{Font}}}}}) \textendash{} (optional) The {\hyperref[\detokenize{pypoints:pypoints.Font}]{\sphinxcrossref{\sphinxcode{\sphinxupquote{Font}}}}} of the point

\item {} 
\sphinxstyleliteralstrong{\sphinxupquote{active}} (\sphinxstyleliteralemphasis{\sphinxupquote{bool}}) \textendash{} (optional) (default \sphinxcode{\sphinxupquote{True}}) If \sphinxcode{\sphinxupquote{active}} is \sphinxcode{\sphinxupquote{False}}, then the point is not registered, and therefore not displayed. Use Point.{\hyperref[\detokenize{pypoints:pypoints.Point.activate}]{\sphinxcrossref{\sphinxcode{\sphinxupquote{activate()}}}}} to register and display the point

\end{itemize}

\item[{Example}] \leavevmode
\end{description}\end{quote}

\begin{sphinxVerbatim}[commandchars=\\\{\}]
\PYG{n}{point} \PYG{o}{=} \PYG{n}{Point}\PYG{p}{(}\PYG{l+s+s2}{\PYGZdq{}}\PYG{l+s+s2}{a}\PYG{l+s+s2}{\PYGZdq{}}\PYG{p}{,} \PYG{l+m+mi}{3}\PYG{p}{,} \PYG{l+m+mi}{5}\PYG{p}{,} \PYG{l+m+mi}{0}\PYG{p}{,} \PYG{n}{font}\PYG{p}{)}
\end{sphinxVerbatim}


\sphinxstrong{See also:}


{\hyperref[\detokenize{pypoints:pypoints.Font}]{\sphinxcrossref{\sphinxcode{\sphinxupquote{Font}}}}}, {\hyperref[\detokenize{pypoints:pypoints.Text}]{\sphinxcrossref{\sphinxcode{\sphinxupquote{Text}}}}}



\begin{sphinxadmonition}{warning}{Warning:}
More than one character on parameter \sphinxcode{\sphinxupquote{char}} will break PyPoints. Please use {\hyperref[\detokenize{pypoints:pypoints.Text}]{\sphinxcrossref{\sphinxcode{\sphinxupquote{Text}}}}} for multi-character points.
\end{sphinxadmonition}
\index{activate() (pypoints.Point method)@\spxentry{activate()}\spxextra{pypoints.Point method}}

\begin{fulllineitems}
\phantomsection\label{\detokenize{pypoints:pypoints.Point.activate}}\pysiglinewithargsret{\sphinxbfcode{\sphinxupquote{activate}}}{}{}
Activate, display, and register the point if \sphinxcode{\sphinxupquote{active}} was \sphinxcode{\sphinxupquote{False}}

\end{fulllineitems}

\index{draw() (pypoints.Point method)@\spxentry{draw()}\spxextra{pypoints.Point method}}

\begin{fulllineitems}
\phantomsection\label{\detokenize{pypoints:pypoints.Point.draw}}\pysiglinewithargsret{\sphinxbfcode{\sphinxupquote{draw}}}{\emph{win}}{}
This method is used by {\hyperref[\detokenize{pypoints:pypoints.PointRegistry}]{\sphinxcrossref{\sphinxcode{\sphinxupquote{PointRegistry}}}}} and \sphinxcode{\sphinxupquote{Run()}}.

\begin{sphinxadmonition}{note}{Note:}
There is no need to call this function
\end{sphinxadmonition}


\sphinxstrong{See also:}


{\hyperref[\detokenize{pypoints:pypoints.PointRegistry}]{\sphinxcrossref{\sphinxcode{\sphinxupquote{PointRegistry}}}}}, \sphinxcode{\sphinxupquote{Run()}}



\end{fulllineitems}

\index{remove() (pypoints.Point method)@\spxentry{remove()}\spxextra{pypoints.Point method}}

\begin{fulllineitems}
\phantomsection\label{\detokenize{pypoints:pypoints.Point.remove}}\pysiglinewithargsret{\sphinxbfcode{\sphinxupquote{remove}}}{\emph{kill=False}}{}
Removes the point from the registry
\begin{quote}\begin{description}
\item[{Parameters}] \leavevmode
\sphinxstyleliteralstrong{\sphinxupquote{kill}} (\sphinxstyleliteralemphasis{\sphinxupquote{bool}}) \textendash{} (optional) (default \sphinxcode{\sphinxupquote{False}}) Deletes the point. Any subsequent calls or references to the point will fail because the point no longer exists.

\end{description}\end{quote}

\begin{sphinxadmonition}{warning}{Warning:}
This action is irreversable
\end{sphinxadmonition}

\end{fulllineitems}


\end{fulllineitems}

\index{PointRegistry (class in pypoints)@\spxentry{PointRegistry}\spxextra{class in pypoints}}

\begin{fulllineitems}
\phantomsection\label{\detokenize{pypoints:pypoints.PointRegistry}}\pysigline{\sphinxbfcode{\sphinxupquote{class }}\sphinxcode{\sphinxupquote{pypoints.}}\sphinxbfcode{\sphinxupquote{PointRegistry}}}
Bases: \sphinxcode{\sphinxupquote{object}}

A PyPoints registry class that is used to register {\hyperref[\detokenize{pypoints:pypoints.Point}]{\sphinxcrossref{\sphinxcode{\sphinxupquote{Point}}}}}

\begin{sphinxadmonition}{warning}{Warning:}
Do not use this class. It is an internal usage class only
\end{sphinxadmonition}


\sphinxstrong{See also:}


{\hyperref[\detokenize{pypoints:pypoints.Point}]{\sphinxcrossref{\sphinxcode{\sphinxupquote{Point}}}}}


\index{register() (pypoints.PointRegistry method)@\spxentry{register()}\spxextra{pypoints.PointRegistry method}}

\begin{fulllineitems}
\phantomsection\label{\detokenize{pypoints:pypoints.PointRegistry.register}}\pysiglinewithargsret{\sphinxbfcode{\sphinxupquote{register}}}{\emph{point}}{}
\end{fulllineitems}

\index{remove() (pypoints.PointRegistry method)@\spxentry{remove()}\spxextra{pypoints.PointRegistry method}}

\begin{fulllineitems}
\phantomsection\label{\detokenize{pypoints:pypoints.PointRegistry.remove}}\pysiglinewithargsret{\sphinxbfcode{\sphinxupquote{remove}}}{\emph{point}}{}
\end{fulllineitems}


\end{fulllineitems}

\index{Shape (class in pypoints)@\spxentry{Shape}\spxextra{class in pypoints}}

\begin{fulllineitems}
\phantomsection\label{\detokenize{pypoints:pypoints.Shape}}\pysiglinewithargsret{\sphinxbfcode{\sphinxupquote{class }}\sphinxcode{\sphinxupquote{pypoints.}}\sphinxbfcode{\sphinxupquote{Shape}}}{\emph{blueprint}, \emph{x}, \emph{y}, \emph{field}}{}
Bases: \sphinxcode{\sphinxupquote{object}}
\index{draw() (pypoints.Shape method)@\spxentry{draw()}\spxextra{pypoints.Shape method}}

\begin{fulllineitems}
\phantomsection\label{\detokenize{pypoints:pypoints.Shape.draw}}\pysiglinewithargsret{\sphinxbfcode{\sphinxupquote{draw}}}{}{}
\end{fulllineitems}


\end{fulllineitems}

\index{Text (class in pypoints)@\spxentry{Text}\spxextra{class in pypoints}}

\begin{fulllineitems}
\phantomsection\label{\detokenize{pypoints:pypoints.Text}}\pysiglinewithargsret{\sphinxbfcode{\sphinxupquote{class }}\sphinxcode{\sphinxupquote{pypoints.}}\sphinxbfcode{\sphinxupquote{Text}}}{\emph{x}, \emph{y}, \emph{text}, \emph{cfield}, \emph{font}, \emph{blueprint=None}}{}
Bases: \sphinxcode{\sphinxupquote{object}}
\index{draw() (pypoints.Text method)@\spxentry{draw()}\spxextra{pypoints.Text method}}

\begin{fulllineitems}
\phantomsection\label{\detokenize{pypoints:pypoints.Text.draw}}\pysiglinewithargsret{\sphinxbfcode{\sphinxupquote{draw}}}{}{}
\end{fulllineitems}

\index{remove() (pypoints.Text method)@\spxentry{remove()}\spxextra{pypoints.Text method}}

\begin{fulllineitems}
\phantomsection\label{\detokenize{pypoints:pypoints.Text.remove}}\pysiglinewithargsret{\sphinxbfcode{\sphinxupquote{remove}}}{\emph{kill=False}}{}
\end{fulllineitems}


\end{fulllineitems}

\index{VLine (class in pypoints)@\spxentry{VLine}\spxextra{class in pypoints}}

\begin{fulllineitems}
\phantomsection\label{\detokenize{pypoints:pypoints.VLine}}\pysiglinewithargsret{\sphinxbfcode{\sphinxupquote{class }}\sphinxcode{\sphinxupquote{pypoints.}}\sphinxbfcode{\sphinxupquote{VLine}}}{\emph{sy}, \emph{ey}, \emph{x}, \emph{char}, \emph{cfield}, \emph{font=None}}{}
Bases: \sphinxcode{\sphinxupquote{object}}

Makes a verticle line
\begin{quote}\begin{description}
\item[{Parameters}] \leavevmode\begin{itemize}
\item {} 
\sphinxstyleliteralstrong{\sphinxupquote{sy}} (\sphinxstyleliteralemphasis{\sphinxupquote{int}}) \textendash{} Start \sphinxcode{\sphinxupquote{y}}

\item {} 
\sphinxstyleliteralstrong{\sphinxupquote{ey}} (\sphinxstyleliteralemphasis{\sphinxupquote{int}}) \textendash{} End \sphinxcode{\sphinxupquote{y}}

\item {} 
\sphinxstyleliteralstrong{\sphinxupquote{x}} (\sphinxstyleliteralemphasis{\sphinxupquote{int}}) \textendash{} \sphinxcode{\sphinxupquote{x}} position

\item {} 
\sphinxstyleliteralstrong{\sphinxupquote{char}} (\sphinxstyleliteralemphasis{\sphinxupquote{string}}) \textendash{} The character that the line consists of

\item {} 
\sphinxstyleliteralstrong{\sphinxupquote{cfield}} (\sphinxstyleliteralemphasis{\sphinxupquote{int}}) \textendash{} The field the line appears

\item {} 
\sphinxstyleliteralstrong{\sphinxupquote{font}} ({\hyperref[\detokenize{pypoints:pypoints.Font}]{\sphinxcrossref{\sphinxcode{\sphinxupquote{Font}}}}}) \textendash{} (optional) The font of the points in the line

\end{itemize}

\item[{Example}] \leavevmode
\end{description}\end{quote}

\begin{sphinxVerbatim}[commandchars=\\\{\}]
\PYG{n}{VLine}\PYG{p}{(}\PYG{l+m+mi}{1}\PYG{p}{,} \PYG{l+m+mi}{7}\PYG{p}{,} \PYG{l+m+mi}{1}\PYG{p}{,} \PYG{l+s+s2}{\PYGZdq{}}\PYG{l+s+s2}{\textbar{}}\PYG{l+s+s2}{\PYGZdq{}}\PYG{p}{,} \PYG{l+m+mi}{0}\PYG{p}{)}
\end{sphinxVerbatim}


\sphinxstrong{See also:}


{\hyperref[\detokenize{pypoints:pypoints.HLine}]{\sphinxcrossref{\sphinxcode{\sphinxupquote{HLine}}}}}


\index{build() (pypoints.VLine method)@\spxentry{build()}\spxextra{pypoints.VLine method}}

\begin{fulllineitems}
\phantomsection\label{\detokenize{pypoints:pypoints.VLine.build}}\pysiglinewithargsret{\sphinxbfcode{\sphinxupquote{build}}}{}{}
There is no need to use this method. It is run automatically

\end{fulllineitems}

\index{remove() (pypoints.VLine method)@\spxentry{remove()}\spxextra{pypoints.VLine method}}

\begin{fulllineitems}
\phantomsection\label{\detokenize{pypoints:pypoints.VLine.remove}}\pysiglinewithargsret{\sphinxbfcode{\sphinxupquote{remove}}}{\emph{kill=False}}{}
Removes all points in the line
\begin{quote}\begin{description}
\item[{Parameters}] \leavevmode
\sphinxstyleliteralstrong{\sphinxupquote{kill}} (\sphinxstyleliteralemphasis{\sphinxupquote{bool}}) \textendash{} (optional) (default \sphinxcode{\sphinxupquote{False}}) Deletes the line. Any subsequent calls or references to the line will fail because the line no longer exists.

\end{description}\end{quote}

\begin{sphinxadmonition}{warning}{Warning:}
This action is irreversable
\end{sphinxadmonition}


\sphinxstrong{See also:}


\sphinxcode{\sphinxupquote{Point.}}{\hyperref[\detokenize{pypoints:pypoints.Point.remove}]{\sphinxcrossref{\sphinxcode{\sphinxupquote{remove()}}}}}



\end{fulllineitems}


\end{fulllineitems}

\index{log() (in module pypoints)@\spxentry{log()}\spxextra{in module pypoints}}

\begin{fulllineitems}
\phantomsection\label{\detokenize{pypoints:pypoints.log}}\pysiglinewithargsret{\sphinxcode{\sphinxupquote{pypoints.}}\sphinxbfcode{\sphinxupquote{log}}}{\emph{txt}}{}
Outputs argument txt to pypointslog.txt
\begin{quote}\begin{description}
\item[{Parameters}] \leavevmode
\sphinxstyleliteralstrong{\sphinxupquote{txt}} (\sphinxstyleliteralemphasis{\sphinxupquote{object}}) \textendash{} The object outputted to the log file

\item[{Example}] \leavevmode
\end{description}\end{quote}

\begin{sphinxVerbatim}[commandchars=\\\{\}]
\PYG{n}{log}\PYG{p}{(}\PYG{l+s+s1}{\PYGZsq{}}\PYG{l+s+s1}{Hello!}\PYG{l+s+s1}{\PYGZsq{}}\PYG{p}{)}

\PYG{n}{log}\PYG{p}{(}\PYG{p}{\PYGZob{}}\PYG{l+s+s1}{\PYGZsq{}}\PYG{l+s+s1}{foo}\PYG{l+s+s1}{\PYGZsq{}}\PYG{p}{:} \PYG{l+s+s1}{\PYGZsq{}}\PYG{l+s+s1}{bar}\PYG{l+s+s1}{\PYGZsq{}}\PYG{p}{\PYGZcb{}}\PYG{p}{)}
\end{sphinxVerbatim}

\end{fulllineitems}

\index{run() (in module pypoints)@\spxentry{run()}\spxextra{in module pypoints}}

\begin{fulllineitems}
\phantomsection\label{\detokenize{pypoints:pypoints.run}}\pysiglinewithargsret{\sphinxcode{\sphinxupquote{pypoints.}}\sphinxbfcode{\sphinxupquote{run}}}{\emph{r}}{}
Runs the “pre” and “run” method from argument r.
These methods are where the program code goes
\begin{quote}\begin{description}
\item[{Parameters}] \leavevmode
\sphinxstyleliteralstrong{\sphinxupquote{r}} (\sphinxstyleliteralemphasis{\sphinxupquote{object}}) \textendash{} The class that is run

\end{description}\end{quote}
\begin{itemize}
\item {} \begin{description}
\item[{r.\sphinxstylestrong{pre}(win)}] \leavevmode
Run before anything else.
Initialize any objects here

\end{description}

\item {} \begin{description}
\item[{r.\sphinxstylestrong{run}(win)}] \leavevmode
Main program code goes here.
Once it finishes, it get ran again

\end{description}

\end{itemize}
\begin{quote}\begin{description}
\item[{Example}] \leavevmode
\end{description}\end{quote}

\begin{sphinxVerbatim}[commandchars=\\\{\}]
\PYG{k}{class} \PYG{n+nc}{program}\PYG{p}{(}\PYG{p}{)}\PYG{p}{:}
    \PYG{k}{def} \PYG{n+nf}{pre}\PYG{p}{(}\PYG{n+nb+bp}{self}\PYG{p}{,} \PYG{n}{win}\PYG{p}{)}\PYG{p}{:}
        \PYG{n+nb+bp}{self}\PYG{o}{.}\PYG{n}{color} \PYG{o}{=} \PYG{n}{Color}\PYG{p}{(}\PYG{n}{green}\PYG{p}{,} \PYG{n}{black}\PYG{p}{)}
        \PYG{n+nb+bp}{self}\PYG{o}{.}\PYG{n}{font} \PYG{o}{=} \PYG{n}{Font}\PYG{p}{(}\PYG{p}{)}
        \PYG{n+nb+bp}{self}\PYG{o}{.}\PYG{n}{point} \PYG{o}{=} \PYG{n}{Point}\PYG{p}{(}\PYG{l+s+s2}{\PYGZdq{}}\PYG{l+s+s2}{a}\PYG{l+s+s2}{\PYGZdq{}}\PYG{p}{,} \PYG{l+m+mi}{0}\PYG{p}{,} \PYG{l+m+mi}{0}\PYG{p}{,} \PYG{l+m+mi}{0}\PYG{p}{)}
    \PYG{k}{def} \PYG{n+nf}{run}\PYG{p}{(}\PYG{n+nb+bp}{self}\PYG{p}{,} \PYG{n}{win}\PYG{p}{)}\PYG{p}{:}
        \PYG{k}{if} \PYG{n}{win}\PYG{o}{.}\PYG{n}{getkey} \PYG{o}{==} \PYG{l+s+s2}{\PYGZdq{}}\PYG{l+s+s2}{c}\PYG{l+s+s2}{\PYGZdq{}}\PYG{p}{:}
            \PYG{k}{return} \PYG{n+nb+bp}{False}
\PYG{n}{r} \PYG{o}{=} \PYG{n}{program}\PYG{p}{(}\PYG{p}{)}
\PYG{n}{run}\PYG{p}{(}\PYG{n}{r}\PYG{p}{)}
\end{sphinxVerbatim}

\end{fulllineitems}

\index{text\_to\_blueprint() (in module pypoints)@\spxentry{text\_to\_blueprint()}\spxextra{in module pypoints}}

\begin{fulllineitems}
\phantomsection\label{\detokenize{pypoints:pypoints.text_to_blueprint}}\pysiglinewithargsret{\sphinxcode{\sphinxupquote{pypoints.}}\sphinxbfcode{\sphinxupquote{text\_to\_blueprint}}}{\emph{txt}, \emph{font}, \emph{file=None}}{}
\end{fulllineitems}



\chapter{PyPoints Tutorials}
\label{\detokenize{tutorial:pypoints-tutorials}}\label{\detokenize{tutorial::doc}}

\section{Blueprints}
\label{\detokenize{blueprint:blueprints}}\label{\detokenize{blueprint::doc}}
PyPoints blueprints are a special thing that allow custom objects, {\hyperref[\detokenize{pypoints:pypoints.MenuBox}]{\sphinxcrossref{\sphinxcode{\sphinxupquote{MenuBox}}}}}es, and {\hyperref[\detokenize{pypoints:pypoints.Text}]{\sphinxcrossref{\sphinxcode{\sphinxupquote{Text}}}}}.


\chapter{Indices and tables}
\label{\detokenize{index:indices-and-tables}}\begin{itemize}
\item {} 
\DUrole{xref,std,std-ref}{genindex}

\item {} 
\DUrole{xref,std,std-ref}{modindex}

\item {} 
\DUrole{xref,std,std-ref}{search}

\end{itemize}


\renewcommand{\indexname}{Python Module Index}
\begin{sphinxtheindex}
\let\bigletter\sphinxstyleindexlettergroup
\bigletter{p}
\item\relax\sphinxstyleindexentry{pypoints}\sphinxstyleindexpageref{pypoints:\detokenize{module-pypoints}}
\end{sphinxtheindex}

\renewcommand{\indexname}{Index}
\printindex
\end{document}